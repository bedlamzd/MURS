%! Author = bedlamzd
%! Date = 28.02.2021

% Preamble
\documentclass[14pt]{extarticle}

%! Author = bedlamzd
%! Date = 16.02.2021

\usepackage{fontspec}
\usepackage{polyglossia}
\defaultfontfeatures{Ligatures=TeX}
\setdefaultlanguage{russian}
\setotherlanguage{english}
\setmainfont{PT Astra Serif}
\newfontfamily{\latinfont}{PT Astra Serif}
\newfontfamily{\cyrillicfont}{PT Astra Serif}
\newfontfamily{\cyrillicfonttt}{FreeMono}

\usepackage{geometry}

\usepackage{mathtools}
\usepackage{amsmath}
\usepackage{amssymb}
\usepackage{amsfonts}
\usepackage{graphicx}
\usepackage{float}
\usepackage{wrapfig}
\usepackage{subcaption}

\geometry{right=20mm}
\geometry{left=20mm}
\geometry{top=20mm}
\geometry{bottom=20mm}

\usepackage{indentfirst}
\usepackage[outputdir=out]{minted}

\usepackage{booktabs}
\usepackage{array}

\newcolumntype{C}{>{$}c<{$}}

\renewcommand{\theFancyVerbLine}{\ttfamily{\normalsize\oldstylenums{\arabic{FancyVerbLine}}}}
\renewcommand{\listingscaption}{Листинг}

\newminted{python}{autogobble, linenos, fontsize=\small}
\newmintinline{python}{fontsize=\small}
\newmintedfile{python}{autogobble, linenos, fontsize=\small, breakanywhere, breaklines}

\newminted{matlab}{autogobble, linenos, fontsize=\small}
\newmintinline{matlab}{fontsize=\small}
\newmintedfile{matlab}{autogobble, linenos, fontsize=\small, breakanywhere, breaklines}

\newminted{octave}{autogobble, linenos, fontsize=\small}
\newmintinline{octave}{fontsize=\small}
\newmintedfile{octave}{autogobble, linenos, fontsize=\small, breakanywhere, breaklines}

%\renewcommand{\thesubsection}{\arabic{subsection}}

\graphicspath{{../img/}}



% Document
\begin{document}
    \begin{titlepage}
    \begin{center}
        \begin{small}
            \textbf{Министерство науки и высшего образования Российской Федерации}

            \vspace{1em}

            ФЕДЕРАЛЬНОЕ ГОСУДАРСТВЕННОЕ АВТОНОМНОЕ ОБРАЗОВАТЕЛЬНОЕ\\
            УЧРЕЖДЕНИЕ ВЫСШЕГО ОБРАЗОВАНИЯ

            \vspace{1em}

            \textbf{<<НАЦИОНАЛЬНЫЙ ИССЛЕДОВАТЕЛЬСКИЙ УНИВЕРСИТЕТ ИТМО>>}
        \end{small}

        \vspace{13ex}

        Практическая работа №1\\
        <<Прямая задача кинематики>>\\
        по дисциплине <<Моделирование и управление робототехническими системами>>
    \end{center}

    \vspace{14em}

    \begin{flushright}
        \noindent
        Выполнил:\\
        студент гр. R41341c\\
        Борисов М. В.

        \vspace{1em}
        Преподаватель:\\
        Каканов М. А.
    \end{flushright}

    \vfill

    \begin{center}
        \large{Санкт-Петербург}\\
        2021 г.\\
    \end{center}
\end{titlepage}


    \section*{Дано}

    \section*{Задание}
    \begin{enumerate}
        \item Выбрать системы координат, связанные со звеньями в соответствии с представлением Денавита-Хартенберга
        \item Выбрать параметры Денавита-Хартенберга
        \item Сформировать матрицы однородного преобразования для каждого из звеньев и рассчитать итоговую матрицу,
        связывающую инерциальную систему координат с системой координат инструмента
        \item Параметризовать матрицу поворота с помощью углов Эйлера
    \end{enumerate}

    \section*{Решение}
    Системы координат выбраны как показано на рисунке~\ref{pic:frames}
    \begin{figure}[H]
        \centering
        \includegraphics[width=0.75\textwidth]{frames.png}
        \caption{Системы координат звеньев манипулятора}
        \label{pic:frames}
    \end{figure}

    Параметры Денавита-Хартенберга приведены в таблице~\ref{tbl:params}.
    Значения параметров $a_i,\ d_i$ были приняты единычными. Поскольку все звенья манипулятора
    вращательные, то углы $\theta_i$ выступают в качестве обобщённых координат и будут выбраны произвольно.
    {
    \renewcommand{\arraystretch}{2}
    \begin{table}[H]
        \centering
        \begin{tabular}{*{5}{c}}\toprule
            Звено, $i$  & $a_i$ & $\alpha_i$        & $d_i$ & $\theta_i$ \\ [1ex] \midrule
            1           & 0     & $\dfrac{\pi}{2}$  & $d_1$ & $\theta_1$ \\ [1ex] \midrule
            2           & $a_2$ & 0                 & 0     & $\theta_2$ \\ [1ex] \midrule
            3           & 0     & $\dfrac{\pi}{2}$  & 0     & $\theta_3 + \dfrac{\pi}{2}$ \\ [1ex] \midrule
            4           & 0     & $-\dfrac{\pi}{2}$ & $d_4$ & $\theta_4$ \\ [1ex] \midrule
            5           & 0     & $\dfrac{\pi}{2}$  & 0     & $\theta_5$ \\ [1ex] \midrule
            6           & 0     & 0                 & $d_6$ & $\theta_6$ \\ [1ex]
            \bottomrule
        \end{tabular}
        \caption{Параметры Денавита-Хартенберга}
        \label{tbl:params}
    \end{table}
    }

    В приложении~\ref{code:fk} приведена функция для решения прямой задачи кинематики. В нём задаются параметры
    манипулятора и производится расчёт. Также в приложении~\ref{code:ht} приведена функция расчёта однородного преобразования.

    В ходе расчётов получаем матрицу
    $T =
    \begin{bmatrix}
        R(q) & p(q)\\
        1 & 1
    \end{bmatrix}$,
    где $R(q)$ --- матрица поворота, $p(q)$ --- координаты системы координат связанной со схватом относительно базы.

    Матрица поворота требует дополнительных преобразований для получения углов Эйлера, что выполняется в конце функции.

    Функция принимает на вход массив поворотов звеньев относительно исходной конфигурации и возвращает
    координаты положения выходного звена манипулятора.
    \begin{matlabcode*}{xleftmargin=\parindent, frame=single}
        >> fk([0.28 0.2 0.1 0.9 0.9 0.9])

        ans =

            2.4620    0.0695    2.1431   -0.6581    0.8647   -0.8527
    \end{matlabcode*}

    \section*{Вывод}
    В ходе работы написана функция решения прямой задачи кинематики.

    \appendix
    \renewcommand{\thesection}{\Asbuk{section}}
    \section{Функция решения прямой задачи кинематики}\label{code:fk}
    \octavefile[frame=single]{../src/fk.m}

    \section{Функция нахождения однородного преобразования}\label{code:ht}
    \octavefile[frame=single]{../src/ht.m}

\end{document}
